\documentclass[margin]{res} % Use the res.cls style

\usepackage{helvet}

\usepackage{textcomp}
\setlength{\textwidth}{5.1in}

\usepackage[utf8]{inputenc}
\usepackage[T2A]{fontenc}
\usepackage[english,russian]{babel}

\begin{document}

\moveleft.5\hoffset\centerline{\large\bf Семён Новиков}
\moveleft\hoffset\vbox{\hrule width\resumewidth height 1pt}\smallskip
\moveleft.5\hoffset\centerline{semka.novikov@gmail.com}
\moveleft.5\hoffset\centerline{+7 (919) 499 74 17}
\moveleft.5\hoffset\centerline{http://www.sdfgh153.ru}

%----------------------------------------------------------------------------------------

\begin{resume}

\section{Вакансия}
Разработчик

\section{Образование}
Пермский Государственный Технический Университет, Специальность АСУ. Специалист, год окончания 2008 

\section{Навыки} 

{\sl Основные языки:} 
Objective-C, C, Ruby

{\sl Второстепенные языки:}
Javascript, Erlang, Smalltalk, C\#, Common Lisp

{\sl Фреймворки и технологии:}
Ruby On Rails, .NET (2.0), jQuery, \LaTeX, REST, Cocoa, xUnit, xSpec, Erlang OTP

{\sl Операционные системы:} 
Mac OS X, Linux, FreeBSD, Windows, Plan 9

{\sl СУБД:}
Sqlite, MySQL, PostgreSQL, CouchDB, Oracle

{\sl Системы контроля версий:} 
git, svn, cvs, mercurial

{\sl Багтракеры:}
Redmine, Github Issues, Mantis
 
\section{Опыт работы}

{\sl iOS разработчик} \hfill сентябрь 2011 - настоящее время \\
Gipis, http://gip.is 
\begin{itemize} \itemsep -2pt
\item Разработал iOS клиент
\end{itemize}
{\sl iOS разработчик} \hfill май 2009 - сентябрь 2011 \\
ООО «Ньюлана», http://nulana.com
\begin{itemize} 
\item Реализовал веб фреймворк и сайт http://nulana.com
\item Принимал участие в разработке компилятора языка Microlisp 
\item Разработал iOS приложения Carmindy, Game Gazing, DJ Sasha
\item Создал и поддерживал инфраструктурные проекты Nulana (средство локализации ПО, бэкенды для Carmindy и Game Gazing, и т. п.)
\end{itemize} 

{\sl Программист} \hfill январь 2008 - май 2009 \\
ООО «Информационные и телекоммуникационные системы»
\begin{itemize}
\item Разработка серверной части файлообменной сети «Херабора» (она правда так называлась)
\item Поддержка партнерской системы
\end{itemize} 

{\sl Программист} \hfill июнь 2006 - январь 2008 \\
ООО «УРАЛСТРОЙПРОЕКТ», http://uralstroyproject.ru
\begin{itemize}
\item Участвовал в разработке системы электронного документооборота
\item Автоматизация работы в AutoCAD
\end{itemize} 

{\sl Программист} \hfill сентябрь 2004 - август 2005 \\
Березниковский Филиал ПГТУ, http://bf.pstu.ac.ru
\begin{itemize}
\item Участвовал в разработке средств автоматизации составления расписаний, средств автоматизации печати дипломов.
\end{itemize} 

\section{Владение языками}
\begin{itemize}
\item Русский: родной
\item Английский: средний уровень (читаю и могу объясниться)
\end{itemize}

\section{Личная информация:}
День рождения: 25.08.1986\\
Женат, детей нет. 

\section{Всякое} 
\begin{itemize}
\item Пару раз в год хожу в походы
\item Плохо играю в Го и на губной гармошке
\item Волонтер-переводчик в http://youversion.com
\end{itemize}

\end{resume}
\end{document}

