\documentclass[13pt]{res} % Use the res.cls style

\usepackage{cmap}
\usepackage[utf8]{inputenc}
\usepackage[english,russian]{babel}
\usepackage[T1]{fontenc}

\renewcommand{\labelitemi}{$-$}

\begin{document}

\moveleft.5\hoffset\centerline{\large\bf Семён Новиков}
\moveleft\hoffset\vbox{\hrule width\resumewidth height 1pt}\smallskip
\moveleft.5\hoffset\centerline{me@sdfgh153.ru}
\moveleft.5\hoffset\centerline{+7 (919) 499 74 17}
\moveleft.5\hoffset\centerline{http://www.sdfgh153.ru}

%----------------------------------------------------------------------------------------

\begin{resume}

\section{Образование}
Пермский Государственный Технический Университет, Специальность АСУ. Специалист, год окончания 2008 

\section{Навыки} 

{\sl Языки программирования} 
\begin{itemize}
\item Objective-C/C, 4 года, промышленная разработка для iOS, Mac OS X, GNUStep
\item Ruby, 2 года, разработка бэкэндов, REST-сервисов
\item Семейство Lisp'ов (Nulana Microlisp, Scheme, Common Lisp), 2 года, промышленная разработка
\item C\#, один год, промышленная разработка
\item Haskell, один год, личные проекты, внутренние средства автоматизации
\item F\#, меньше года, личные проекты
\end{itemize}

{\sl Прочее по специальности}
\begin{itemize}
\item Есть опыт администрирования UNIX (Linux, FreeBSD, Mac OS X, Plan 9)
\item Опыт работы с СУБД sqlite, MySQL, PostgreSQL, Oracle
\item Знаю git, svn, mercurial
\item В общих чертах знаю Javascript/Typescript
\end{itemize} 

{\sl Opensource проекты и активность}
\begin{itemize}
\item http://code.sdfgh153.ru
\item http://github.com/semka
\end{itemize}

\section{Опыт работы}

{\sl Ведущий iOS разработчик} \hfill февраль 2013 - настоящее время \\
ЗАО <<Прогноз>>, http://prognoz.ru 
\begin{itemize} 
\item Разрабатываю iPad приложения
\end{itemize}

{\sl iOS разработчик} \hfill сентябрь 2011 - февраль 2013 \\
Gipis, http://gip.is 
\begin{itemize} 
\item Разработал iOS клиент
\end{itemize}

{\sl iOS разработчик} \hfill май 2009 - сентябрь 2011 \\
ООО «Ньюлана», http://nulana.com
\begin{itemize} 
\item Реализовал веб фреймворк и сайт http://nulana.com
\item Принимал участие в разработке компилятора языка Microlisp 
\item Разработал iOS приложения Carmindy, Game Gazing, DJ Sasha
\item Создал и поддерживал инфраструктурные проекты Nulana (средство локализации ПО, бэкенды для Carmindy и Game Gazing, и т. п.)
\end{itemize} 

{\sl Программист} \hfill январь 2008 - май 2009 \\
ООО «Информационные и телекоммуникационные системы»
\begin{itemize}
\item Разработка серверной части файлообменной сети «Херабора» (она правда так называлась)
\item Поддержка партнерской системы
\end{itemize} 

{\sl Программист} \hfill июнь 2006 - январь 2008 \\
ООО «УРАЛСТРОЙПРОЕКТ», http://uralstroyproject.ru
\begin{itemize}
\item Участвовал в разработке системы электронного документооборота
\item Автоматизация работы в AutoCAD
\end{itemize} 

{\sl Программист} \hfill сентябрь 2004 - август 2005 \\
Березниковский Филиал ПГТУ, http://bf.pstu.ac.ru
\begin{itemize}
\item Участвовал в разработке средств автоматизации составления расписаний, средств автоматизации печати дипломов.
\end{itemize} 

\section{Владение языками}
\begin{itemize}
\item Русский: родной
\item Английский: средний уровень (читаю и могу объясниться)
\end{itemize}

\section{Личная информация}
День рождения: 25.08.1986

\end{resume}
\end{document}

